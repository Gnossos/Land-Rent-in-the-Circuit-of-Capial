\begin{abstract}
Despite claims that ground rent is ``the primary conceptual vehicle''
for incorporating geography into political economy \citep[136]{christophers2016forreal},
ground rent and the macroeconomic logic of the capitalist space economy
are out of fashion as theoretical concerns in contemporary geographic
political economy \citep{ward2016virtual,Peck_2016_Macroeconomic,sheppard2011geographical}.
Consequently, although ground rent played an obvious role triggering
the socioeconomic crisis of the past decade, economic geography has
contributed remarkably little to either our understanding of the macroeconomics
of the crisis or policies to deal with its causes.

This paper introduces recent heterodox macroeconomic theory into geographic
political economy and spatiality into formal heterodox macroeconomic
theory. It does so by combining ground rent with the ``second-generation''
synthesis of Marxian, Keynesian, and institutionalist macroeconomic
political economy (\citealt{Goldstein_2009_Heterodox}; also see \citealt{godley2012monetary}
and \citealt{keen2011adynamic}, \citeyear{keen2013amonetary}).

Most previous work on ground rent focuses on the mechanisms and institutions
through which rent is extracted, but this paper takes ground rent
as given and focuses on its macroeconomics. The paper first ``clears
the air'' by discussing heterodox economics and geographical political
economy, methodology, value, and ground rent. It then develops several
exploratory, abductive simulation models, incrementally incorporating
different heterodox economic elements and forms of ground rent. The
models build on Marx's circuit of capital but add post-Keynesian and
institutionalist features, and they employ twenty-first century innovations,
such as Foley's \citeyearpar{foley1982realization,foley1986understanding,foley2013profitrates}
dynamic disequilibrium models of the circuit and recent reinterpretations
Marxian value theory \citep{freeman2010trendsin}. The goal is to
develop a better understanding of the macroeconomics of ground rent
that can inform research and policy in both heterodox economics and
geographical political economy.
\end{abstract}
\begin{comment}
\bibliographystyle{plain}
\bibliography{\string"/Users/marsh/OneDrive/Documents/Research-King Mac/Political Economy/Land Rent in the Circuit of Capial/References\string"}
\end{comment}

\printbibliography[title={References}]
