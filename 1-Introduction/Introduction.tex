\begin{comment}
With myMemoir, don't put title, etc. in LyX.
\end{comment}


\section{Introduction}

The U.S. housing system was at the center of the crisis of the early
twenty-first century.\footnote{This name for the crisis comes from \citet{Dumenil_2012_Crisis} and
seems most appropriate. Currently there is no consensus on the timing
or nature of the crisis, and its visible form has evolved through
several phases: increased defaults and foreclosures in the U.S. housing
market (2006), liquidity crisis and credit crunch (2007), failure
and distress of major financial institutions (2007-8), drastic government
intervention (2007-13), a global recession of epic proportions (2007
- 2009+), Europe's sovereign debt and austerity imbroglio (2009 -
present), and Brexit and the rise of right-wing populist movements
in Europe and the U.S. (c. 2009 - present).

A major goal of this paper is to better integrate heterodox economics
and economic geography. In addition to \citeauthor{Dumenil_2012_Crisis},
some examples of heterodox economics analyses of the crisis with strong
geographic relevance include: \citet{foley2009theanatomy}, \citet{hudson2012thebubble},
\citet{lucarelli2012financialization}, \citet{palan2013thefinancial},
\citet{pettifor2013thenext}, and Kotz (\citeyear{kotz2013thecurrent},
\citeyear{kotz2015capitalism}).} In the bubble leading up to the crisis, house prices increased 188\%
between 1997 and 2006 \citep{levitin2012explaining}. Because land
prices are a major component of housing prices, generally believed
to reflect capitalized streams of land rent, and the supply of land
is relatively limited in the short term, land rent is the prime candidate
for the lion's share of the house-price inflation.\footnote{A common misunderstanding of land is that, as a use value, it cannot
be created but is instead an inherently limited ``scarce resource''
\citep[806]{elden2010landterrain}. I address this issue below.} Indeed, this was the case (see Figure ??). So if land is ``the economic
fulcrum of the capitalist remaking of geography'' \citep[1]{ward2016virtual}
and ``the primary conceptual vehicle by which geography ... can be
imported into otherwise aspatial political-economic thinking'' \citep[136]{christophers2016forreal},
one might expect economic geographers not only to have much to say
in the wake of the crisis about the role of land rent, but also to
have been intensely studying land rent before the crisis and perhaps
even to have been among those who predicted it. Unfortunately, this
was not the case \citep{ward2016virtual}, and this lacunae appears
to be yet another ``boat'' that economic geography missed (\citealt{dicken2004geographers};
also see \citealt{taylor2012islandlife}).\footnote{Seeking to debunk government officials' and mainstream economists'
claim that the crisis was unforeseen and unforeseeable, \citet[7-9]{bezemer2009noone}identified
authors who met four criteria. They must have: not only predicted
the crisis, but also explained how they arrived at their prediction;
linked the prediction about a real estate crash to wider economic
impacts; publicly released their analysis; and given a time frame
for their predictions. These criteria yielded a list of twelve persons,
including ``academics, government advisers, consultants, investors,
stock market commentators and one graduate student.'' Not one economic
geographer is on the list.}

This should not surprise anyone familiar with the current state of
economic geography.\footnote{Strictly speaking, ``economic geography'' refers to a sub-discipline
of geography, but here I use it more broadly to include certain strains
of urban studies outside the discipline of geography proper. I believe
the description of the current state of economic geography presented
here applies outside the discipline of geography as much as inside.} Since the late 1980's, the field has become increasingly fragmented
and disconnected \citep{scott07economic,barnes2010nothing}. Rapid,
geographically discontinuous change became common in the global space
economy, making the economic prospects of what had previously been
the economically secure and dominant ``core'' of the global economy
increasingly tenuous and uncertain. This motivated economic geographers
to look for (and often claim they had discovered) the next new thing
that could save homeland capitalism, and economic geography went through
a corresponding series of faddish ``turns'' (technological, flexible,
cultural, relational, evolutionary, etc.). It became preoccupied with
``the churn and chop of capitalist restructuring,'' but paying little
or no attention to ``the structural dynamics of that churn and chop''
\citet[11]{peck2016macroeconomic}. Its gaze settled on the local
and regional, the micro and meso, the new and novel. This became so
characteristics of the field that in his response to \citeauthor{hudson2016risingpowers}'s
\citeyearpar{hudson2016risingpowers} account of ongoing transformations
in the global economy, \citet{Peck_2016_Macroeconomic} stresses how
uncommon ``macroscopic questions'' like the ones \citeauthor{hudson2016risingpowers}
addresses have become in contemporary economic geography.\footnote{\label{fn:macroeconomic}\citeauthor{peck2016macroeconomic}'s otherwise
excellent commentary has two unfortunate characteristics. The term
``macroeconomic'' in the title evokes a sense of how both mainstream
and heterodox economists practice macroeconomics within the discipline
of economics. For the most part, they study the structures of whole
economies, the articulation of the component parts of such structures,
and their dynamics. While \citeauthor{hudson2016risingpowers}'s paper
presumes such elements, they are mainly implicit. Instead, the paper
is an interpretive description of ongoing changes in the global economy.
So the ``macroeconomic'' label could be misleading.

Second, in discussing the novelty of \citeauthor{hudson2016risingpowers}'s
intervention, \citeauthor{Peck_2016_Macroeconomic} briefly recounts
the history of economic geography since the 1970's. In doing so he
metaphorically borrows a practice from computer software development,
calling the prevalent academic practice during the period from around
1970 to the mid-1990's ``economic geography 1.0'' and the (multiple,
fragmented, isolated) practices during the subsequent period, ``economic
geography 2.0.'' But in computer software such nomenclature typically
connotes improvement in subsequent generations, in the sense that
they correct earlier errors (fix bugs) and enhance capabilities. But
neither characteristic is self-evident in economic geography as practiced
since the 1990's. In fact, a good case can be made that the second
phase is retrograde. } Furthermore, as the field became fragmented, debates between devotees
of different approaches occasionally became acrimonious, and this
combined with the tendency for inbreeding within approaches, not only
to further fragment the field, but also to turn different approaches
into self-referential, self-reinforcing ``intellectual solitudes''
\citep{barnes2010nothing} or ``tribes'' \citep{peck2012economic}.
Although no comprehensive study of the methodology underpinning the
field's current research practice exists, it is hard to resist the
suspicion that eclectic and ad-hoc is common. Finally, much as the
flexible specialization literature asserts firms avoid making major
capital investments in turbulent economic environments \citep{piore1984thesecond},
in the turbulent environment of academic Anglophone economic geography,
researchers are prone to choose topics requiring smaller, less risky
investments.\footnote{This may be especially true for younger scholars. Studying the structural
dynamics of capitalism requires some familiarity with mainstream macroeconomics
and its heterodox counterpoints, as well as enough mathematics both
to make one's way through this often math-laden literature and to
conduct independent research. \citet[490]{scott07economic} notes
that scholars trained during economic geography's quantitative phase
often transitioned later to doing political-economic research on broad
structural dynamics. Even though they rejected the positivism and
apologetics of ``spatial science,'' their knowledge of the mainstream
literature and their quantitative training served them well for this
task. But after twenty-five years of fragmentation, things have changed.
\citet[119]{peck2012economic} suggests economic geography has experienced
a ``generational shift towards more qualitative approaches.'' If
so, then younger scholars today may be ill-equipped for studying capitalism's
macro-structural dynamics and choose to confine themselves to topics
amenable to qualitative research at micro- and meso- levels..} Rigor can be a casualty of this proclivity to make small investments
combined with the rush to join the latest fad, and observers have
noted that the quality of recent research often leaves much to be
desired \citep{markusen1fuzzyconcepts,martin2001rethinking}.\footnote{On top of this, the field's ventures into such things as poststructuralism
and postmodernism introduced widely varying standards of what legitimately
counts as high-quality research. In such an environment, what is one
person's rigor is another's rigor mortis. }

But the bigger point is that economic geography today is nowhere near
the ideal of a scientific community, joined together around the common
goal of developing a shared understanding of the world by building
on a common research program, focusing on a consensual set of questions,
and finding ways to adjudicate disagreements. In this context, the
likelihood that even a subset of contemporary economic geographers
would identify potential threats to the global economic order, consistently
monitor such threats, and based on such research warn about the order's
immanent implosion is virtually nil. Mainstream economics may be autistic
??, but economic geography suffers from attention deficit disorder.

Nonetheless, despite its disorder and fragmentation, certain unifying
threads run through contemporary economic geography. These include
the shared beliefs that capitalism is neither universal nor superior
to other forms of economic organization; that capitalism is necessarily
subject to crises and generally in a state of disequilibrium; that
there is no such thing as a separate, self-governing ``economy,''
but instead ``economic'' aspects of society are always embedded
in a wider social web of culture, institutions, ecology, and ``nature'';
that spatiality matters in that space cannot be reduced to a Cartesian
container but instead is intrinsic to human life, and the social,
historical, and spatial aspects of human societies constitute and
determine each other in a socio-spatial dialectic; that society is
not an undifferentiated collection of individuals, much less one with
common goals and bound together by social contracts or other forms
of mutual consent, but instead consists of social relations of unequal
power along various axes of domination/subordination; and that concrete
research on actual economies in specific places is particularly important.
(See\citealt{sheppard2011geographical,sheppard2012thelong,peck2016macroeconomic};
and \citealt{soja1980thesociospatial}.) These shared beliefs situate
economic geography in opposition to mainstream economic orthodoxy
thereby qualifying economic geography as heterodox relative to mainstream
economics \citep{Peck_2015_Navigating,sheppard00000heterodox}.\footnote{\citeauthor{Peck_2015_Navigating}sees such unorthodox beliefs and
practices as a weak qualification for being heterodox and points to
economic geography's pluralism as stronger qualification. But on its
own pluralism per se does not qualify a set of practices as heterodox,
and in fact some authors claim mainstream economics is already pluralist
and becoming more so. See below.} But these beliefs are very general and therefore leave room for great
variation in the practice of economic geography, so these threads
tying economic geography together are quite loose.

Marxism is possibly a stronger thread. David Harvey's brilliant work
\citep[esp.][]{harvey2006thelimits} has been remarkably influential
in the field, and he has steadfastly promoted and advanced the Marxist
tradition in geography. \citet{sheppard2012thelong} maintain that
``anyone seeking to take economic geography in a new and different
direction still feels compelled to rationalize their position relative
to his {[}Harvey's{]}, either to declare affinities to, or distance
themselves from, Harvey and Marxism.'' \citeauthor{sheppard2011geographical}(\citeyear{sheppard2011geographical},
320; \citealt[11]{sheppard2012thelong}) thus describes the field
as ``haunted by Marx.'' \footnote{Sheppard cites \citet{derrida25specters} as the source of this term.
But my own reading of Derrida's book is that its title refers to the
``spectre'' that Marx and Engels described as haunting Europe at
the beginning of the \emph{Communist Manifesto} (i.e., communism),
and Derrida's book (which originally was a lecture) is about using
post-structuralism as a tool to identify something in today's world
that could serve a similar role as a radical liberation movement.
In relation to this, Derrida identifies ten ``plagues'' of today's
global capitalism. But in keeping with this obviously religious allusion,
much of Derrida's book deals with similar religious and supernatural
allusions in Marx. To me the most memorable and dramatic of these
(and the most relevant to this paper) is the image of ``an enchanted,
perverted, topsy-turvy world, in which Monsieur le Capital and Madame
la Terre do their ghost-walking as social characters and at the same
time directly as mere things'' \citep[830]{marx1967capital}. Nonetheless,
the practice of describing Derrida as himself being ``haunted''
by Marx in that he (Dirrida) had to deal with Marx, and describing
other authors the same way, perhaps with some deliberate irony, is
not uncommon. For example, without reference to Derrida, \citet{walker2004thespectre}
uses similar imagery specifically with respect to David Harvey's influence
on geography.} This metaphor is very apropos. Many contemporary economic geographers
treat not only Marx but the entire Marxist tradition more like a corpse
than as something living and evolving. Hence, in post-1990 economic
geography one finds almost zero references to new developments within
Marxian economics, such as new, non-equilibrium, non-dual interpretations
of the labor theory of value \citep{freeman2010trendsin} or the copious
literature rethinking Marx's theory of money in relation to production
\citep{graziani2003themonetary,wray1999theories}; conversely, today
very few economic geographers \textendash{} even, or especially, those
whose work is very much in a Marxist vein \textendash{} make much
effort to present their work as extending or modifying received Marxian
political economy. Even around 1990, when the then-dominant ``structuralist''
Marxian geographical political economy came under increasing criticism
\citep{scott07economic}, few critics sought to reconcile their criticisms
with the received approach by correcting or augmenting it. This was
despite the fact that the critics were often sympathetic in that they
shared with the ``structuralists'' the inclination to reject spatial
science's technocratic impulse in favor of a broader project of human
liberation, and their criticisms of the received approach often maintained
it was incomplete and too restrictive rather than incorrect .\footnote{Incomplete by ignoring or downplaying class conflict, gender, human
subjectivity and agency, non-capitalist elements in existing economies,
and so on. Of course, an approach that is incomplete very likely will
be unable to explain empirical situations on its own, and the causal
efficacy of what it claims to know will depend on what elements it
omits. So in this sense that structural logic was (mis)applied to
certain situations, the approach was incorrect. Besides this, perhaps
the main substantive criticism of the approach per se is that it was
insufficiently open. See \citet[490 - 491]{scott07economic}. Of course,
this is a gross generalization, but overall I think the characterization
is correct.}

This is not to deny that some economic geographers, most notably David
Harvey, continue to work in an explicitly Marxist frame, or that ideas
from Marxist political economy permeate much of contemporary economic
geography. Instead, the point is that economic geography has mostly
shifted its focus to ``the meso- and micro-scales'' \citep[700]{jones2016geographies},
which in itself is not problematic except insofar as the larger, macro-level
disappears and these other scales appear as context-independent, free
from any larger forces, and all there is.\footnote{The term ``macro'' is ambiguous in that it is used to refer to large
geographic scales (e.g., global or continental) or to entire economies,
which could be at much smaller scales (e.g., Chile or Sweden). Also
see note \ref{fn:macroeconomic}.} Furthermore, without explicit consideration of the macro ``purist''
geographical political economy and heterodox macroeconomics, one has
the sense that economic geography runs the danger of borrowing theoretical
concepts in an eclectic, ad hoc manner, with little concern for coherence,
much less a coherent, cumulative research program. It may well be
that under certain conditions scientific fields benefit from being
open to extreme diversity and pluralism, but surely scientific progress
implies that at some point a given field coalesces around shared acceptance
of a specific paradigm \emph{within which} there may still be diversity,
or there even may be multiple paradigms within a live-and-let-live
arrangement, but there is not an almost random, diverse pluralism,
with no shared coherence whatsoever.\footnote{\citet{feyerabend1993against} be damned.}

Turning to land rent, one sees a similar pattern.\footnote{\citet{ward2016theshitty}distinguish between land and ground rent.
The former includes payments for the use of not only land but also
of buildings and other improvements associated with particular parcels
of land. Ground rent, on the other hand, consists of payments for
the use of land net of payments for the use of fixed capital on the
land. Generally, a parcel of land's location is reflected in land-rent,
but the portion of the payment attributable to location is usually
treated as solely due to the location of the land even though improvements
are located there too. Therefore, in urban contexts, location is thought
to be captured by ground rent, and since the literature emphasizes
location, ``ground rent'' plays the central role. Nonetheless, this
terminology is not used universally, and improvements on land also
capture locational advantages (e.g., a gas station near a freeway
offramp); furthermore, in a very real sense improvements ``make''
location (e.g. a skyscraper office building helps define ``downtown.'')
In other words, the relation between improvements and location is
dialectical, and land rent not so easily split apart.} After earlier periods of consensus during the 1970s and transition
during the early 1980s, the study of land rent ruptured in the late
1980s \citep{haila1990thetheory}. In one respect, it broke into two
camps: one ``ideographic'' and emphasizing empirical research on
specific situations, and the other, ``nomothetic'' camp emphasizing
more general theorization. Along another axis of fragmentation, some
authors emphasized land rent's role in accumulation, while others
emphasized its coordinating role. \citet[276]{haila1990thetheory}
also identifies a contradiction between research motivated by contemporary
issues but often falling back on ``canonical dogma'' and sees the
field as being at a crossroads, either continuing in this vein or
developing theory ``in an unbiased manner.'' Since the study of
land rent is subject to the same currents as the larger, parent field
of economic geography, it is unsurprising that this bifurcation occurred
around the same time as when centrifugal forces and fragmentation
became pronounced throughout economic geography.

However, unlike the rest of economic geography, which shattered into
so many fragments, the study of land rent entered a ``magic roundabout''
of circular reasoning \citep{kerr1996thetheory} that effectively
functioned as a theoretical and methodological cul-de-sac, followed
by disinterest and neglect of the subject \citep{park2014landrent,ward2016theshitty,christophers2016forreal}.
A good indicator of this disappearance from economic geography is
land rent's virtual absence from contemporary textbooks and surveys
of the field at all levels, from introductory to advanced \citep{mackinnon2011introduction,wood2012economic,barnes2012thewileyblackwell}.\footnote{\citet{anderson2012economic} textbook is an interesting exception.
Unlike most other contemporary textbooks, its center is spatial science,
and it has barely a hint of geographical political economy. Even so,
its sources concerning land rent all seem several decades old. This
is difficult to determine with certainty because the book has a large
list of references, it generally does not cite sources in the body
of the text. } And this neglect is not confined to economic geography: similar patterns
of disinterest and neglect for land rent, as well as most forms of
rentier income, can be found in such cognate fields as heterodox economics
(Hudson and Bezemer 2012) and critical organization studies (Bohm,
Land, and Beverugen 2012).

Various authors attempt to explain this neglect. For example, although
he issues the caveat that a full answer to why land has been ``theoretically
side-lined ... is beyond the scope of this article'' \citep[136]{christophers2016forreal},
\citet{christophers2016forreal} claims the treatment of land as \emph{fictitious
}\textendash{} fictitious capital in Marx and fictitious commodity
in Polanyi \textendash{} is a major reason for the neglect of land
and land rent: ``is it any wonder that theorists have focused on
the real stuff and made the fictitious a secondary consideration?
... How, after all, can something fictitious be pivotal?'' \citep[137, original emphasis]{christophers2016forreal}.
This argument is unpersuasive not only as an explanation for the neglect
of land and land rent, but also as a general assessment of the role
of ``fictitious'' entities in political economy.

Regarding Polanyi, \citet[134]{christophers2016forreal} himself notes
that economic geographers have only shown substantial interest in
Polanyi for about a decade, and the earliest example of such interest
\citeauthor{christophers2016forreal} offers is \citet{prudham2012knockon}.
Since declining interest in land rent started roughly two-and-a-half
decades ago \citep{park2014landrent,ward2016theshitty}, comparatively
recent interest in Polanyi can hardly be responsible for the decline.
But \citet{christophers2016forreal}seems to consider the practice
of treating land as something fictitious renders it peripheral to
political economy perpetually, in which case there would be no decline
to explain; fictitiousness could have been the cause all along, and
recent interest in Polonyi merely rediscovered a different strain
of a long-standing malady. But this argument runs aground on the rocks
of the periods of consensus, transition, and even rupture that \citet{haila1990thetheory}
so aptly documents: disinterest in land and land-rent only became
common after 1990. Finally, Polanyi's basic assumptions are flawed.
He labels as ``fictitious'' anything that has not been produced
for sale, and therefore puts labor, land, and money in this category
(Polanyi 75??). But recent heterodox political economy \citep{bertocco2007thecharacteristics,graziani2003themonetary}
argues that banks do ``create'' money for ``sale'' (i.e., for
lending), and below I discuss how land is produced for sale.

Evidence regarding Marx also undermines Chrisotphers' argument. If
Marx's treatment of ``fictitious capital'' caused disinterest in
all things fictitious, particularly land rent, then one would expect
periods with high interest in Marx to be marked by disinterest in
land rent. This is clearly not the case. During the two decades from
1970 through 1990 interest in Marxian political economy among economic
geographers was at perhaps its highest point \citep[488 - 491]{scott07economic};
yet this period was also one of intense interest in land rent \citep{haila1990thetheory}.
Furthermore, false beliefs are a central theme \textendash{} arguably
the central theme \textendash{} in Marx's project. He wrote about
many false beliefs and practices based on such fictions: ideology,
religion, commodity fetishism, etc. In fact, explaining how social
systems necessarily generate particular fictions (i.e., false beliefs)
is one of the hallmarks of critical social science \citep{sayer1critical}.
On top of this, Marx himself originally (in 1857) planned to devote
an entire book of \emph{Capital} to land rent, and although he repeatedly
changed the outline for his opus over the next decade, as late as
the 1870s he was still working on the subject \citep[12 -13, 55 - 56, 22 ]{rodolsky1977themaking}.
If land rent is of only peripheral interest because it is fictitious
capital, why would Marx devote so much attention to it? Indeed, Marx
at times put land rent on equal footing with capital itself, with
both cast as fictitious beings seemingly hiding behind masks: ``Monsieur
le Capital and Madame la Terre'' coexist in ``an enchanted, perverted,
topsy-turvy world'' where they ``do their ghost-walking as social
characters and at the same time directly as mere things'' \citep[830]{marx1967capital}.
In short, merely labeling land rent as ``fictitious capital'' is
hardly sufficient grounds to ignore it.

More fundamentally, the premise of this argument conflates three concepts
that, in Marxist theory, are quite distinct: land, land rent, and
fictitious capital. The first two parallel Marx's treatment of the
use-value and exchange-value in \emph{Capital}. Inserting ``land''
where he describes the use-values of commodities makes perfect sense:
land is ``an assemblage of many properties, and may therefore be
of use in various ways'' with ``socially-recognised standards of
measure,'' and ``the utility of a thing {[}i.e., land{]} makes it
a use-value,'' which ``being limited by the physical properties''
of any parcel of land ``has no existence apart from'' that parcel
\citep[35 - 36]{marx1967capital2}. In other words, ``land'' refers
to a use-value, or more precisely something that has use-value.

The parallel with exchange-value is a bit more difficult to establish.
``Exchange-value, at first sight, presents itself as a quantitative
relation, as the proportion in which values in use of one sort are
exchanged for those of another'' \citep[36]{marx1967capital}. Land
rent is the price of accessing the use-value of land and is measured
in some medium \textendash{} typically money \textendash{} that allows
access to land to be exchanged for ownership of or access to other
commodities. So if a parcel of land rents for \$1,000/mo. and a pair
of shoes sells for \$100, then a month's use of the parcel exchanges
for the equivalent of ten pairs of shoes. In this sense, land rent
is completely analogous to the exchange value of most commodities.
But Marx goes two steps further, first maintaining that ``the exchange-values
of commodities must be capable of being expressed in terms of something
common to them all'' \citep[37]{marx1967capital}. Then, deducing
that the only possible ``something'' is ``the labour-time socially
necessary \ldots{} to produce an article under the normal conditions
of production, and with the average degree of skill and intensity
prevalent at the time'' \citep[39]{marx1967capital2}. Since he,
like so many others, mistakenly assumes that humans cannot produce
land, a parcel of land cannot have exchange-value, in the strict sense
that the ``substance'' of exchange value is the socially necessary
time required to produce it.\footnote{Below I discuss the production of land.}
But differential rent is the value of the savings in labor time on
a parcel with above-average productivity over that of parcels with
productivity typical of ``the normal conditions of production,''
so differential rent's substance is the same as exchange-value's,
socially necessary labor time. And absolute rent is the quantum of
social product, measured in value \textendash{} i.e., socially necessary
labor time \textendash{} that landowners are able to appropriate through
their power to deny or to permit access to land and therefore space,
something absolutely necessary for all human existence. So both forms
of rent are a portion of the social product measured in human labor
time. Before interest in land rent declined, theorists studying rent
spilled much ink over these different sources of rent, but they paid
very little attention to the \emph{total} rent. As \citet[74, emphasis in original]{kerr1996thetheory}
points out, ``Marx never suggested that the complex interaction of
DRI and DRII\footnote{Differential rent types I and II.} made it
difficult to distinguish who should get what. The \emph{total} rent
was just that, the \emph{return to landed property}.'' So while land
rent is not necessarily the exchange value \emph{of} land, strictly
speaking with ``exchange value'' meaning the socially necessary
labor to produce it, land-rent is nonetheless the exchange value \emph{appropriated}
by virtue of the power vested through land ownership.

Finally, the fact that the price of land as exchange value is a \emph{mixture}
of the socially necessary labor time to \emph{produce} land \emph{plus}
the \emph{exchange-value appropriated as land-rent} through the power
of landed property does not automatically qualify land-rent as ``fictitious
capital,'' although admittedly this requires selective interpretation
of Marx's ``sketchy'' treatment of the subject \citep[xv]{harvey2006thelimits}. 

\citet[97]{harvey2006thelimits}

and serves to establis, 

``land \emph{rent}'' is the exchange value of land

{*}{*}{*}{*}

xxxxx

This latter period had several dimensions. Emphasizing the social
relations of landed property and its multiple forms in contemporary
capitalism, some scholars rejected the possibility of a general theory
of rent and instead emphasized ``ideographic'' empirical research
on specific situations; opposing this, others emphasized ``nomothetic''
research and searched ``for generalizations and a general theory''
\citep[293]{haila1990thetheory}.\footnote{Haila's excellent review article is a landmark in the history of research
on land rent. Nonetheless, it has some serious issues. One is its
use of empiricist meta-theoretical categories, such as nomothetic
generalization \citep{willer1973systematic}, to (mis)characterize
conceptualization at different levels of abstraction, as is common
in realist, particularly Marxist, social science (see below). This
is compounded by characterizing ``ideographic'' research as ``historical''
and ``nomothetic'' as a search for general \textendash{} and therefore
implicitly ahistorical \textendash{} laws. Instead, abstractions in
critical-realist social research in general, and Marxist research
in particular, generally are historical, and comprehension of even
the most specific, concrete situation generally requires high-level
abstractions.

Another issue is the opposition between ``accumulation'' and ``coordination,''
which \citep[276]{haila1990thetheory} subsumes under ``the economic
role of rent.'' But these are dialectical aspects of a single system
rather than opposites. The pace, nature, and geography of accumulation
depend on how the system is coordinated, and the structure of the
system, how it regulates itself or not, the role of institutions \textendash{}
in other words, the character of its overall means of coordination
\textendash{} is vital to the process of accumulation. Indeed, this
is the point of the present essay.

Furthermore, the ``economic'' role of land rent is not limited to
accumulation and coordination. It plays important roles in production,
distribution, consumption, reproduction, }

xxx

However, unlike economic geography as a whole, the study of rent This
led many scholars to turn away from the subject, resulting not only
in a general neglect of land rent itself \citep{ward2016theshitty},
but also in a ``stalled'' literature dealing with the more general
topic of the role of land under capitalism \citep{christophers2016forreal,park2014landrent}.\footnote{\citet{christophers2016forreal} attributes this stall to the fact
that both Marx and Polanyi label land rent as a form of ``fictitious
capital,'' which causes scholars to take land rent and land less
seriously. For example, he speculates \citet[144]{christophers2016forreal}
``perhaps if land were conceived as real as opposed to fictitious
capital we would not so readily forget in the first place {[}that
land-related financial vehicles are ultimately tied to private property
and ``solid, physical assets.'' For many reasons, I do not find
this argument convincing and instead would argue that there are many
other reasons why land rent has been neglected. But my goal here is
not to explain this neglect, nonetheless I return to this subject
below, in the discussion of land rent.}

\begin{comment}
\bibliographystyle{plain}
\bibliography{References}
\end{comment}

\printbibliography[title={References}]
